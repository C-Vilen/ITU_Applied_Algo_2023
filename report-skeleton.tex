\documentclass[12pt, a4paper]{article}
\usepackage[utf8]{inputenc}
\usepackage{graphicx}

% Update this information to reflect yourself
\title{Assignment 1: 3SUM}
\author{Christian Vilen}
\date{2022-09-20}

\begin{document}
\maketitle

\section{Introduction}

Introduce the problem you are addressing here, and give a very brief
overview of what you are about to report on.

\section{Implementation}

Explain briefly what you have implemented and how. This might also be
an appropriate place to explain how you verified the correctness of
your implementation.

\subsection{Hardware and Software specification}
All of the experiments has been executed on the following equipment:
\begin{itemize}
\item Computer: MacBook Pro, 14-inch 2021.
\item Hardware: Apple M1 Pro, 16gb ram.
\item OS Version: Mac OS X Ventura 13.5.1
\item Java: 17.0.5 2022-10-18 LTS
\item Gradle: 7.6
\item Kotlin: 1.7.10
\end{itemize}

\subsection{Tests of ThreeSum Algorithms}
The implementation of the three different ThreeSum algorithms has been tested with various JUnit tests to verify their correctness. This includes multiple variable cases of 0, 1 and more triplets summing to 0 or not.

\section{Experiments}

Explain what kind of experiments you run. Also tell what kind of
hardware and software were used to run the experiments.

Present your results numerically in a table, and also visually as a
plot. Remember to refer to the tables and figures in your text; it is
not sufficient to only include them, but they should also be made part
of the prose. Like so: here we refer to Table~\ref{tbl:resultscubic}.

\begin{table}[h]
  \begin{center}
  \caption{Write a caption that tells what this table is about.}
  \label{tbl:resultscubic}
  % Uncomment the line below to include the automatically generated
  % table from the file.
  % \input{threesum_cubic_tabular.tex}
  \end{center}
\end{table}
https://www.overleaf.com/project/64fabec8a0a055e063505cfa
\begin{figure}[h]
  \begin{center}
    % uncomment the line below to include the plot that you
    % automatically generated.
    % \includegraphics[width=\textwidth]{plot_cubic_vs_hashmap.pdf}
    \caption{Write a descriptive caption here.}
    \label{fig:runtimes}
  \end{center}
\end{figure}

\end{document}